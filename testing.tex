\section{Tests and Test Status} \label{sec:testing}
Based roughly on \secref{sec:definition} there is a section here for each test we perform.
We consider since ComCam will not go on sky readiness may be demonstrated using AuxTel and/or Startracker data.

Any notebooks used for testing this will be in the notebook folder of the github repo containing this document.

\subsection{Data transferred and ingested in butler and access to data}
For a given observing day, verify after the end of the day that each raw image (identified by sequence number and detector) is available in the \texttt{/repo/embargo} Butler repository by retrieving it from both a development node (\texttt{rubin-devl}) and the USDF RSP (notebook and terminal).
Measure the delays between completion of readout for each image and the ingestion time recorded in the Butler Registry.
Verify that the 95th(TBD) percentile is less than 1 minute(TBD).

\textbf{This has been sucessfull.}

\subsubsection{Test execution}
\paragraph{Summit}
In the repo of this document \texttt{RTN-053-summit.ipynb} (\secref{sec:summitnb}) give the count of raws for a given night and lists detector and exposureid for the first and last.

\begin{verbatim}
There are 592 raw in collection: ['LATISS/raw/all'] in: /repo/embargo for exposure.day_obs = 20230511
0 2023051100001
0 2023051100599
\end{verbatim}

\paragraph{USDF}
In the repo of this document \texttt{RTN-053-USDF.ipynb} (\secref{sec:usdfnb}) gives the count of raws for the same night and lists detector and exposureid for the first and last.

\begin{verbatim}
There are 592 raw in collection: ['LATISS/raw/all'] in: /repo/embargo for exposure.day_obs = 20230511
0 2023051100001
0 2023051100599
\end{verbatim}

\paragraph {Ingest}
Also in the USDF notebook we get the ingest time and subtract the shutter close time to calculate the
delay to ingest.
\begin{verbatim}
Got Metadata for 592 exposures
The min time was 0.19347863333343263 max time was 0.5866245333332909 minutes
Time between shutter close and ingest mean:0.23482382218469838, std:0.018058551157743862, var:0.0003261112699168522 minutes
\end{verbatim}

\subsection{LFA replication to USDF }
Sample LargeFileObjectAvailable events from the USDF EFD (which will be verified as corresponding to the Summit EFD below).
Retrieve the corresponding LFA objects from the USDF LFA object store.
\subsubsection{Test execution}
In the same notebooks  listed earlier ther eis a section 3.2
\paragraph{Summit}
\begin{verbatim}
Got 173 Large files availble since 2023-04-20 at summit
\end{verbatim}

\paragraph{USDF}
\begin{verbatim}
Got 173 Large files availble since 2023-04-20 at usdf
Access result[92] https://s3.cp.lsst.org/rubinobs-lfa-cp/Electrometer:201/fits/2023/05/02/Electrometer:201_fits_2023-05-03T00:32:23.907.fits
Access result[44] https://s3.cp.lsst.org/rubinobs-lfa-cp/Electrometer:201/fits/2023/05/02/Electrometer:201_fits_2023-05-02T23:37:56.243.fits
Access result[83] https://s3.cp.lsst.org/rubinobs-lfa-cp/Electrometer:201/fits/2023/05/02/Electrometer:201_fits_2023-05-03T00:22:18.627.fits
Access result[112] https://s3.cp.lsst.org/rubinobs-lfa-cp/Electrometer:201/fits/2023/05/02/Electrometer:201_fits_2023-05-03T00:54:52.978.fits
Access result[146] https://s3.cp.lsst.org/rubinobs-lfa-cp/Electrometer:201/fits/2023/05/02/Electrometer:201_fits_2023-05-03T01:33:03.891.fits
\end{verbatim}

\subsection{Automated prompt processing}
Verify that images taken with the ScriptQueue that have nextVisit events issued can be processed by a pipeline.
The pipeline will execute at least one step of single-frame instrument signature removal.
Verification will consist of retrieving the data products from the \texttt{/repo/embargo} Butler repository.

\subsection{EFD data available }
Choose at least 5 EFD topics at random from those available at the Summit.
Choose at least 20 messages per topic from those available at the Summit, including 10 from the previous observing day and 10 from past history.
Verify that all 100 messages are present in the USDF EFD.

Choose 5 more topics that have graphs in the Summit Chronograf.
Verify that the graph shown for each topic for a specified time range is the same as that displayed by the USDF Chronograf.
