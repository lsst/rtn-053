\documentclass[OPS,authoryear,toc]{lsstdoc}
\input{meta}
\def\milestone{ USDF ready for ComCam processing}

% Package imports go here.
\usepackage{pdfpages}
\usepackage{caption}
\usepackage{subcaption}
\usepackage{setspace}
% Local commands go here.

\usepackage{listings}
\lstset{
basicstyle=\small\ttfamily,
columns=flexible,
breaklines=true
}
%If you want glossaries
%\input{aglossary.tex}
%\makeglossaries

\title{L2 - \milestone}

% Optional subtitle
% \setDocSubtitle{A subtitle}

\author{%
William O'Mullane, K.T. Lim, Heinrich Reinking
}

\setDocRef{RTN-053}
\setDocUpstreamLocation{\url{https://github.com/lsst/rtn-053}}

\date{\vcsDate}

% Optional: name of the document's curator
% \setDocCurator{The Curator of this Document}

\setDocAbstract{%
This is a short description of   what we will have in place for the Milestone " L2 - USDF ready for ComCam processing"  and the tests performed to confirm it is achieved.
}

% Change history defined here.
% Order: oldest first.
% Fields: VERSION, DATE, DESCRIPTION, OWNER NAME.
% See LPM-51 for version number policy.
\setDocChangeRecord{%
  \addtohist{0.1}{2023-02-26}{Pre test.}{W. O'Mullane, K.T. Lim}
  \addtohist{0.2}{2023-05-29}{Summit and USDF Chronograph plots}{H. Reinking}
  \addtohist{1.0}{2023-05-30}{Tidy remove draft}{W. O'Mullane}
}


\begin{document}

% Create the title page.
\maketitle
% Frequently for a technote we do not want a title page  uncomment this to remove the title page and changelog.
% use \mkshorttitle to remove the extra pages


\section{Introduction}

This document serves as a description of the milestone {\it\milestone} as well as a report on testing to achieve the milestone.
The operations milestone ticket for this is  \jira{PREOPS-1379}.
The description is in \secref{sec:definition} while the test status is in \secref{sec:testing}.

\section{Milestone description} \label{sec:definition}

This milestone is about USDF being ready for ComCam which will not now go on sky but it still takes flats etc which we have been processing.

At a broad level this means getting ComCam and ancillary data from the summit to USDF and having it accessible to users.

This entails USDF :

\begin{itemize}
\item data auto-transferred from summit to USDF; ingested in butler
\item LFA replication to USDF (Ceph -> Ceph to start)
\item automated prompt processing
\item EFD data available at USDF
\item access to data from batch/RSP for staff and commissioners.
\end{itemize}



\section{Tests and Test Status} \label{sec:testing}
Based roughly on \secref{sec:definition} there is a section here for each test we perform.
We consider since ComCam will not go on sky readiness may be demonstrated using AuxTel and/or Startracker data.

Any notebooks used for testing this will be in the notebook folder of the github repo containing this document.

\subsection{Data transferred and ingested in butler and access to data}
For a given observing day, verify after the end of the day that each raw image (identified by sequence number and detector) is available in the \texttt{/repo/embargo} Butler repository by retrieving it from both a development node (\texttt{rubin-devl}) and the USDF RSP (notebook and terminal).
Measure the delays between completion of readout for each image and the ingestion time recorded in the Butler Registry.
Verify that the 95th(TBD) percentile is less than 1 minute(TBD).

\subsubsection{Test execution}
As may be seen from the executions below data is replicated but the ingest time is too slow.
The tardiness is not actually a USDF issue though.
\paragraph{Summit}
In the repo of this document \texttt{RTN-053-summit.ipynb} give the count of ras for a given night and lists detector and exposureid for the first and last.
\begin{verbatim}
There are 592 raw in collection: ['LATISS/raw/all'] in: /repo/embargo for exposure.day_obs = 20230511
0 2023051100001
0 2023051100599
\end{verbatim}

\paragraph{USDF}
In the repo of this document \texttt{RTN-053-USDF.ipynb} gives the count of raws for the same night and lists detector and exposureid for the first and last.

\begin{verbatim}
\end{verbatim}

\paragraph {Ingest}
Also in the USDF notebook.
\begin{verbatim}
There are 592 raw in collection: ['LATISS/raw/all'] in: /repo/embargo for exposure.day_obs = 20230511
0 2023051100001
0 2023051100599
\end{verbatim}

\subsection{LFA replication to USDF }
Sample LargeFileObjectAvailable events from the USDF EFD (which will be verified as corresponding to the Summit EFD below).
Retrieve the corresponding LFA objects from the USDF LFA object store.

\subsection{Automated prompt processing}
Verify that images taken with the ScriptQueue that have nextVisit events issued can be processed by a pipeline.
The pipeline will execute at least one step of single-frame instrument signature removal.
Verification will consist of retrieving the data products from the \texttt{/repo/embargo} Butler repository.

\subsection{EFD data available }
Choose at least 5 EFD topics at random from those available at the Summit.
Choose at least 20 messages per topic from those available at the Summit, including 10 from the previous observing day and 10 from past history.
Verify that all 100 messages are present in the USDF EFD.

Choose 5 more topics that have graphs in the Summit Chronograf.
Verify that the graph shown for each topic for a specified time range is the same as that displayed by the USDF Chronograf.




\appendix

\includepdf[pages=1,pagecommand={\section{Summit Notebook \label{sec:summitnb}}}]{RTN-053-summit}
\includepdf[pages=2-]{RTN-053-summit}
\includepdf[pages=1,pagecommand={\section{USDF Notebook \label{sec:usdfnb}}}]{RTN-053-USDF}
\includepdf[pages=2-]{RTN-053-USDF}
% Include all the relevant bib files.
% https://lsst-texmf.lsst.io/lsstdoc.html#bibliographies
\section{References} \label{sec:bib}
\renewcommand{\refname}{} % Suppress default Bibliography section
\bibliography{local,lsst,lsst-dm,refs_ads,refs,books}

% Make sure lsst-texmf/bin/generateAcronyms.py is in your path
\section{Acronyms} \label{sec:acronyms}
\addtocounter{table}{-1}
\begin{longtable}{p{0.145\textwidth}p{0.8\textwidth}}\hline
\textbf{Acronym} & \textbf{Description}  \\\hline

ComCam & The commissioning camera is a single-raft, 9-CCD camera that will be installed in LSST during commissioning, before the final camera is ready. \\\hline
EFD & Engineering and Facility Database \\\hline
L2 & Lens 2 \\\hline
LATISS & LSST Atmospheric Transmission Imager and Slitless Spectrograph \\\hline
LFA & Large File Annex \\\hline
OPS & Operations \\\hline
RSP & Rubin Science Platform \\\hline
RTN & Rubin Technical Note \\\hline
TBD & To Be Defined (Determined) \\\hline
USDF & United States Data Facility \\\hline
\end{longtable}

% If you want glossary uncomment below -- comment out the two lines above
%\printglossaries





\end{document}
