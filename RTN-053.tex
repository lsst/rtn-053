\documentclass[OPS,authoryear,toc]{lsstdoc}
\input{meta}
\def\milestone{ USDF ready for ComCam processing}

% Package imports go here.
\usepackage{pdfpages}
\usepackage{caption}
\usepackage{subcaption}
\usepackage{setspace}
% Local commands go here.

\usepackage{listings}
\lstset{
basicstyle=\small\ttfamily,
columns=flexible,
breaklines=true
}
%If you want glossaries
%\input{aglossary.tex}
%\makeglossaries

\title{L2 - \milestone}

% Optional subtitle
% \setDocSubtitle{A subtitle}

\author{%
William O'Mullane, K.T. Lim, Heinrich Reinking
}

\setDocRef{RTN-053}
\setDocUpstreamLocation{\url{https://github.com/lsst/rtn-053}}

\date{\vcsDate}

% Optional: name of the document's curator
% \setDocCurator{The Curator of this Document}

\setDocAbstract{%
This is a short description of   what we will have in place for the Milestone " L2 - USDF ready for ComCam processing"  and the tests performed to confirm it is achieved.
}

% Change history defined here.
% Order: oldest first.
% Fields: VERSION, DATE, DESCRIPTION, OWNER NAME.
% See LPM-51 for version number policy.
\setDocChangeRecord{%
  \addtohist{0.1}{2023-02-26}{Pre test.}{W. O'Mullane, K.T. Lim}
  \addtohist{0.2}{2023-05-29}{Summit and USDF Chronograph plots}{H. Reinking}
  \addtohist{1.0}{2023-05-30}{Tidy remove draft}{W. O'Mullane}
}


\begin{document}

% Create the title page.
\maketitle
% Frequently for a technote we do not want a title page  uncomment this to remove the title page and changelog.
% use \mkshorttitle to remove the extra pages

\input{body}

\appendix

\includepdf[pages=1,pagecommand={\section{Summit Notebook \label{sec:summitnb}}}]{RTN-053-summit}
\includepdf[pages=2-]{RTN-053-summit}
\includepdf[pages=1,pagecommand={\section{USDF Notebook \label{sec:usdfnb}}}]{RTN-053-USDF}
\includepdf[pages=2-]{RTN-053-USDF}
% Include all the relevant bib files.
% https://lsst-texmf.lsst.io/lsstdoc.html#bibliographies
\section{References} \label{sec:bib}
\renewcommand{\refname}{} % Suppress default Bibliography section
\bibliography{local,lsst,lsst-dm,refs_ads,refs,books}

% Make sure lsst-texmf/bin/generateAcronyms.py is in your path
\section{Acronyms} \label{sec:acronyms}
\addtocounter{table}{-1}
\begin{longtable}{p{0.145\textwidth}p{0.8\textwidth}}\hline
\textbf{Acronym} & \textbf{Description}  \\\hline

ComCam & The commissioning camera is a single-raft, 9-CCD camera that will be installed in LSST during commissioning, before the final camera is ready. \\\hline
EFD & Engineering and Facility Database \\\hline
FITS & Flexible Image Transport System \\\hline
GIS & Global Interlock System \\\hline
L2 & Lens 2 \\\hline
LATISS & LSST Atmospheric Transmission Imager and Slitless Spectrograph \\\hline
LFA & Large File Annex \\\hline
MTM1M3 & Main Telescope M1M3 \\\hline
MTM2 & Main Telescope Secondary Mirror \\\hline
OPS & Operations \\\hline
RSP & Rubin Science Platform \\\hline
RTN & Rubin Technical Note \\\hline
TBD & To Be Defined (Determined) \\\hline
USDF & United States Data Facility \\\hline
\end{longtable}

% If you want glossary uncomment below -- comment out the two lines above
%\printglossaries





\end{document}
